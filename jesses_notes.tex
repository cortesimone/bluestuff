\documentclass[a4paper,10pt]{article} 
%\documentclass[a4paper,10pt]{scrartcl}

\usepackage[utf8]{inputenc}
\usepackage{emerald}
\usepackage[T1]{fontenc}

\title{Jesse's notes} 
\author{} 
\date{}



\begin{document} 
\maketitle

\section*{Ingredients:} 
\begin{itemize}  
  \item 1/2 cup $H_{2}0$ as solvent  
  \item 3/4
  cup of agent: ( $H-(C=O)-(CHOH)_{5}-H$ )  
  \item 2 cups of solute: granulated $C_{12}H_{22}O_{11}$
  (chemically composed of  $H-(C=O)-(CHOH)_{5}-H$ and an isomer of that, obviously
  also $C_{6}H_{12}O_{6}$ )  
  \item 2 tea spoons of awesoming extract of $C_{6}H_{8}O_{7}$   
  \item 1
  drop of $C_{37}H_{34}N_{2}Na_{2}O_{9}S_{3}$ (also called E133) for colouring  
  \item lab thermometer
\end{itemize}





\section*{Process}

Line a baking sheet with foil and spray the foil with nonstick cooking spray.

In a medium saucepan, combine the $H_{2}0$, agent, and the solute. Place the pan over
medium-high heat, and stir until the sugar dissolves. Bring the mixture to a
boil, then stop stirring and brush down the sides with a wet pastry brush to
prevent sugar crystals from forming. Insert the lab thermometer.
\\
\\
\normalfont\ECFAugie
yo, bitch, obviously when you boil our thing with water, the water also boils and
gets evaporated, and our thing gets concentrated. Mr White says that "the
proportion of solute in the solvent gets higher". If it is more concentrated, it
will harden, and become useless. So the important thing here is the temperature,
which makes the water evaporate, not the time we spend doing it. We need to
measure the temperature, yo!

also, yo, remember, if our thing, $C_{12}H_{22}O_{11}$, is more concentrated, it will cristalize. We need
an "agent", says Mr. White, to prevent early cristalization. The molecules of our product match 
like a puzzle, and we need to prevent those mixing between them, to prevent early
cristalization. The agent to add can be:   
\begin{itemize}
 \item $CH_{3}(CH_{2})_{2}COOH$ : with this gets creamy. So, tasty but useless, yo
 \item  $C_{6}H_{12}O_{6}$ (cetose type) and isomer of our agent: with this also works
 \item  $C_{6}H_{12}O_{6}$ with structure $H-(C=O)-(CHOH)_{5}-H$ : with this works. it's gonna be our agent, yo!\\
\end{itemize}
      
\normalfont


Continue to cook the mixture without stirring until it reads 413.15ºK on the
thermometer. Do not overcook it, or it will start to take on a gold color and
the final product might turn green! Once at 413.15ºK, take the pan off the heat
and let it sit for a few moments, until bubbles stop breaking on the surface.
Add the extract of $C_{6}H_{8}O_{7}$ and a drop of $C_{37}H_{34}N_{2}Na_{2}O_{9}S_{3}$ coloring, and stir
everything together.

Pour all onto the prepared baking sheet and tilt it so that it runs into a thin
layer. Let it set completely at room temperature.

Once set, break it into small pieces by hand. Or alternatively, place the pieces
in a large zip-top bag and smash them with a rolling pin until they are crushed.
Store the product in an airtight container at room temperature.


\normalfont\ECFAugie
\begin{center}
 science, bitch!
\end{center}

yo, problems:
\begin{itemize}
 \item The product turns out Green: this is the \#1 problem people have with this
  recipe. There are 2 possible culprits: overcooking the product, and using the
  wrong colouring. 
  \begin{enumerate}
   \item Overcooking.  If allowed to cook to a high temperature, it
  will eventually turn yellow, then amber, then brown. If your mixture is already
  a yellow or brown color when you’re adding the colouring drops, chances are you
  have overcooked it. 
  \item  Wrong colouring. Well, if you do this, then your are a
  douche, bitch!
  \end{enumerate}

  \item The product Is Too Soft/Is Sticky/Doesn’t Set Well: It has not been cooked
enough. This recipe really, truly should be measured using a lab thermometer.
  \item If the product is initially fine but quickly becomes sticky, humidity is
probably the problem. keep in mind the hygroscopy of the product, yo! Keep it as
dry as possible!
\end{itemize}
\normalfont

\newpage
\section*{keys:} 
\begin{itemize}  
    \item 1/2 cup $H_{2}0$ as solvent = water  
    \item 3/4 cup of agent: $H-(C=O)-(CHOH)_{5}-H$ = glucose = corn syrup  
    \item 2 cups of solute:  granulated sugar $C_{12}H_{22}O_{11}$ = saccharose (table sugar). On the recipe we
    say its composed of fructose and glucose.  
    \item 2 tea spoons of awesoming
    extract, C6H8O7 = citric acid flavouring extract, or any flavor you might use in
    baking or candy making: vanilla extract, mint extract.. If you find flavoring
    oils used for candy, those are a lot more concentrated so only use about a
    quarter of the recommended amount if you’re using an oil instead of an extract.
    \item 1 drop of $C_{37}H_{34}N_{2}Na_{2}O_{9}S_{3}$ (also called E133) = blue food colorant  
    \item product = candy  
    \item hard product = hard candy/rock candy  
    \item $C_{6}H_{12}O_{6}$ (cetose type) = fructose  
    \item $CH_{3}(CH_{2})_{2}COOH$ = butyric acid = butter  
    \item $C_{6}H_{8}O_{7}$ = citric acid  
    \item lab thermometer = sugar thermometer 
\end{itemize}


\section*{references:}
\begin{enumerate}
  \item http://www.thekitchn.com/pantry-staples-diy-cane-
sugar-131934 
  \item http://www.sugarhero.com/2012/07/blue-crystal-meth-rock-candy-for-
breaking-bad/
  \item http://www.gastronomiavegana.org/el-laboratorio/la-ciencia-del-
caramelo/
  \item wikipedia for chemistry formulae
\end{enumerate}


 
 


\end{document}
